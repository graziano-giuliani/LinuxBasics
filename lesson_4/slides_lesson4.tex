\documentclass[unknownkeysallowed, 10pt, a4 paper, handout]{beamer}

% Custom beamer theme
\usepackage{../style/beamerthemeCustom}
\newcommand{\HRule}{\rule{\linewidth}{0.5mm}}   %FOR TITLEPAGE

\usepackage{changepage}       % adjustwidth
\usepackage{upquote}
\usepackage{color}
\usepackage{fancyvrb}

\setlength\parskip{0.3cm}

\definecolor{darkgreen}{RGB}{0,100,0}
\newcommand{\green}[1]{\textbf{\textcolor{darkgreen}{#1}}}
\newcommand{\red}[1]{\textbf{\textcolor{red}{#1}}}

\newcommand{\focus}[1]{\textbf{\textcolor{red}{#1}}}
\newcommand{\expire}[1]{\textbf{\textcolor{green}{#1}}}
\newcommand{\ra}{$\longrightarrow$ }
\newcommand{\lra}{$\longleftrightarrow$ }

\newcommand{\code}[1]{\colorbox{black}{\color{green}\texttt{#1}}}
\newcommand{\codeine}[1]{\colorbox{black}{\color{green}\BUseVerbatim{#1}}}

% Command to create two side-by-side minipages
\newcommand{\sidebyside}[5]{
  \begin{minipage}{#1\textwidth}
    #2
  \end{minipage} #3 \begin{minipage}{#4\textwidth}
    #5
  \end{minipage}
}

\title[Linux Programming]{ICTP DP Linux Basic Course - Shell Script}
\subtitle{ESP Students - First Semester}
\author[Graziano Giuliani]{Graziano Giuliani \\ \focus{ggiulian@ictp.it}}
\institute[ICTP]{The Abdus Salam International Centre for Theoretical Physics}
\date[\today]{ICTP Diploma Program \\ \today}

\begin{document}

\begin{frame}
  \titlepage
\end{frame}


\begin{frame}[label=outline]
  \frametitle{Course Outline \footnotemark}
  \framesubtitle{Daily program}
  \begin{itemize}
    \item \expire{UNIX/Linux}
    \item \expire{Programming on Linux}
    \item \expire{Text file manipulation}
    \item \focus{Basic BASH and Python}
      \begin{enumerate}
        \item BASH variables
        \item BASH programming
        \item Python programming
      \end{enumerate}
    \item Basic Fortran
  \end{itemize}

  \vspace{6mm}

  Slides: \\ \code{http://tinyurl.com/2jsvfbd6}
  \vspace{4mm} \\
  or the \LaTeX \ source on GitHub: \\
  \code{https://github.com/graziano-giuliani/LinuxBasics}

  \footnotetext[1]{Course created in 2019 with Adriano Angelone, now LPTMC-FR}

\end{frame}


\begin{frame}
  \frametitle{Shell Scripting}
  \framesubtitle{From REPL to Programming Language}

  \begin{alertblock}{Why Shell Scripting}
    Reusing multiple times the same command sequence? \\
      Write the sequence in a script and create a new command.
  \end{alertblock}
  \begin{block}{Elements of a Programming Language}
    \begin{itemize}
      \item \focus{Execution flow}: Entry point and return point
      \item \focus{Input/Output}: Interaction
      \item \focus{Variables}: Data representation and storage
      \item \focus{Conditionals}: Branch execution on condition on variables
      \item \focus{Loops}: Repeat code until condition on variables met
    \end{itemize}
  \end{block}
  \begin{exampleblock}{}
    Bash shell is a complete programming language.
  \end{exampleblock}
\end{frame}


\begin{frame}
  \frametitle{Variables}
  \framesubtitle{Name and Value}
  \begin{block}{}
    Variables contain data. They have a label (\focus{Name}) and a
      content (\focus{Value}). \\
      In bash, variables \focus{have no type}
    \begin{itemize}
      \item Everything is a string
      \item Arithmetic sometimes possible on integer values
    \end{itemize}
  \end{block}

  \begin{center}
    \includegraphics[width=0.72\textwidth]{pics/jars.jpg}
  \end{center}
\end{frame}


\begin{frame}[fragile=singleslide]
  \frametitle{Variables}
  \framesubtitle{Assign value to name and expand name into value}

  \begin{exampleblock}{Assignment}
    Use the \code{=} character between the variable name and its value content.
    \begin{itemize}
      \begin{SaveVerbatim}{myverb}
name="value"
      \end{SaveVerbatim}
      \item \codeine{myverb}
      \begin{SaveVerbatim}{myverb}
message="A whole string"
      \end{SaveVerbatim}
      \item \codeine{myverb}
      \begin{SaveVerbatim}{myverb}
number=3
      \end{SaveVerbatim}
      \item \codeine{myverb}
    \end{itemize}
  \end{exampleblock}

  \begin{exampleblock}{Expansion}
    Prepend the \code{\$} character to the variable name.
    \begin{itemize}
      \begin{SaveVerbatim}{myverb}
b=$number
c=$b
      \end{SaveVerbatim}
      \item \codeine{myverb}
      \begin{SaveVerbatim}{myverb}
part1="Hello"
part2="World"
whole="${part1} ${part2}!"
      \end{SaveVerbatim}
      \item \codeine{myverb}
    \end{itemize}
  \end{exampleblock}
\end{frame}

\begin{frame}[fragile=singleslide]
  \frametitle{Basic I/O}
  \framesubtitle{echo, read and how to catch program output}

  \small{
  \begin{exampleblock}{print}
    Use the \code{echo} command. \\
    \begin{SaveVerbatim}{myverb}
message="Welcome to my program!"
result=42
echo $message: The answer is $result
    \end{SaveVerbatim}
    \codeine{myverb}
  \end{exampleblock}

  \begin{exampleblock}{read}
      Use the read command, eventually using the \code{-p} (prompt) option. \\
    \begin{SaveVerbatim}{myverb}
read -p "Please enter your name: " user_name
read -p "Please enter your Age : " user_age
echo ${user_name} age is ${user_age}.
    \end{SaveVerbatim}
    \codeine{myverb}
  \end{exampleblock}

  \begin{exampleblock}{Subshell}
    Output from programs can be stored in variables:\\
    \begin{SaveVerbatim}{myverb}
listing=$(ls)
echo $listing
    \end{SaveVerbatim}
    \codeine{myverb}
  \end{exampleblock}
  }
\end{frame}


\begin{frame}
  \frametitle{Variables}
  \framesubtitle{Shell environment variables}

  Every shell has built-in variables assigned at startup. \\
    Their names and values can be listed using the \code{env} command.
  \begin{exampleblock}{Environment variables}
    \begin{itemize}
      \item \code{USER} : User name
      \item \code{HOME} : User home folder path
      \item \code{PATH} : Search path for user binary programs
    \end{itemize}
    To set an environment variable, use \code{export}: \\
    \code{export LC\_ALL="C"}
  \end{exampleblock}

  \begin{alertblock}{}
    Try the \code{env} program in your shell to see the environment variables.\\
    \focus{Can you sort their values by name?} \\
    Try changing the value of \code{LC\_ALL} to the value \code{"C"} and
    run \code{ls}. \\
    Set it back to the value \code{"en\_US.UTF-8"}. Run \code{ls} again. \\
    \focus{Examine what changes.}
  \end{alertblock}
\end{frame}

\begin{frame}[fragile=singleslide]
  \frametitle{Variables}
  \framesubtitle{Command line arguments as variables in script}

  \focus{Script arguments are variables}

  \begin{exampleblock}{Scripts}
    \begin{verbatim}
#!/bin/bash
echo "Script name = $0"
echo "Number of arguments = $#"
echo "First argument = $1"
echo "Second argument = $2"
echo "All arguments = $@"
    \end{verbatim}
  \end{exampleblock}

  \begin{alertblock}{}
    Create a simple script with the above content and run it: \\
      \code{bash test\_script.sh a1 a2 a3 a4 a5} \\
    Can you foretell the output?
  \end{alertblock}
\end{frame}


\begin{frame}
  \frametitle{Conditionals}
  \framesubtitle{Basic structure: test command success}

  \begin{block}{}
    \code{if <test> ; then <action 1> ; else <action 2> ; fi}
  \end{block}

  \begin{exampleblock}{Test command result output}
    \begin{columns}
      \begin{column}{0.60\textwidth}
        \begin{center}
          \includegraphics[width=0.8\textwidth]{pics/conditional_1.png}
        \end{center}
      \end{column}
      \begin{column}{0.40\textwidth}
          True is \code{0} \\
          False is any value not \code{0}
      \end{column}
    \end{columns}
  \end{exampleblock}
\end{frame}


\begin{frame}
  \frametitle{Conditionals}
  \framesubtitle{test conditions on file and variables}

  \begin{block}{}
    \begin{columns}
      \begin{column}{0.60\textwidth}
        \small{
          \begin{itemize}
            \item \code{[ -f <file> ]}: file exists
            \item \code{[ -d <directory> ]}: dir exists
            \item \code{[<string1> ==/!= <string2>]}
            \item \code{[<int1> <operator> <int2>]} \\
            \code{<operator>} can be:
            \begin{itemize}
              \item \code{-eq/ne}: equal/not equal
              \item \code{-lt/le}: less/less or equal
              \item \code{-gt/ge}: greater/greater or equal
            \end{itemize}
          \end{itemize}
          \begin{itemize}
            \item \code{[... -a ...]} is AND
            \item \code{[... -o ...]} is OR
            \item \code{[! ...]} is NOT
          \end{itemize}
        }
      \end{column}
      \begin{column}{0.40\textwidth}
        \includegraphics[scale=0.23]{pics/conditional_2.png}
      \end{column}
    \end{columns}
  \end{block}
\end{frame}


\begin{frame}
  \frametitle{Iteration}
  \framesubtitle{Perform actions several times}

  \begin{block}{}
    \code{for <variable> in <range> ; do <action> ; done}
  \end{block}

  \begin{exampleblock}{}
    \begin{columns}
      \begin{column}{0.60\textwidth}
        \begin{itemize}
          \item \code{variable} is created for the loop and cannot be
             reached outside of the loop. \\
             \code{<action>} can use \code{<variable>}
          \item \code{range} is a sequence or a regexp to be exapnded
        \end{itemize}
      \end{column}
      \begin{column}{0.40\textwidth}
        \includegraphics[width=0.90\textwidth]{pics/for.png}
      \end{column}
    \end{columns}
  \end{exampleblock}
\end{frame}

\begin{frame}
  \frametitle{Exercise}
  \framesubtitle{Put it all together: File processing}

  \begin{alertblock}{This will be a long exercise}
    Complex programs are done in steps: start simple, then add complexity.
   \end{alertblock}

  \begin{exampleblock}{}
    Assume that we have an external program, \code{binaver}, to be used to
    average data of a single targer column from a column based file. \\
    It requires information about the file, its row and column total number,
    and only accepts one file at a time. \\
    The program can be run with the following syntax: \\

    \code{binaver -r<rows> -c<columns> -k<target\_column> <file>}\\

    We will write a script (\focus{wrapper}) which:

    \begin{itemize}
      \item Counts automatically rows and columns
      \item Excludes commented lines
      \item Checks the integrity of the file
      \item Applies \code{binaver} to multiple files
    \end{itemize}
  \end{exampleblock}
\end{frame}

\begin{frame}
  \frametitle{Step}
  \framesubtitle{Counting rows}

  \begin{exampleblock}{Task 1}
    Write a script which assigns to a variable the number of rows of a file.
  \end{exampleblock}

  \begin{block}{}
    Rows can be counted with the \code{wc -l <file>} command, but
    the command however outputs \code{<line number> <file>}\\
    Rows can also be counted using \code{awk}
  \end{block}

  \begin{alertblock}{Hint}
    Use command substitution and a pipeline with \code{awk} or 
    \code{wc} and \code{cut}. \\
    Assume the file to be the first command line argument and
    remember how you access it.
  \end{alertblock}
\end{frame}


\begin{frame}[fragile=singleslide]
  \frametitle{Solution}
  \framesubtitle{Counting rows}

  \begin{exampleblock}{Using \code{awk}}
    \begin{verbatim}
#!/bin/bash

file=$1
nrows=$(awk 'END{print NR}' $file)
# Debug
echo "File $file has $nrows lines"
    \end{verbatim}
  \end{exampleblock}

  \begin{exampleblock}{Using \code{wc} and \code{cut}}
    \begin{verbatim}
#!/bin/bash

file=$1
nrows=$(wc -l $file | cut -d " " -f 1)
# Debug
echo "File $file has $nrows lines"
    \end{verbatim}
  \end{exampleblock}
\end{frame}


\begin{frame}
  \frametitle{Step}
  \framesubtitle{Counting columns}

  \begin{exampleblock}{Task 2}
    Extend the previous script, assigning to a variable the number of columns
    of the same file.
  \end{exampleblock}

  \begin{alertblock}{Hint}
    One way uses \code{awk}, its internal variables, and \code{sort}.\\
    For now, assume that all rows have the same number of columns.
  \end{alertblock}

  \begin{exampleblock}{Task 3}
    Modify the script so that it proceeds only if all rows
    have the same number of columns
  \end{exampleblock}

  \begin{alertblock}{Hint}
    If you did it as suggested before, you could use the previous bit of
    the script here. This checks the \focus{integrity} of the file
  \end{alertblock}
\end{frame}


\begin{frame}[fragile=singleslide]
  \frametitle{Solution}
  \framesubtitle{Counting columns}

  \begin{exampleblock}{Rows and columns both}
    \begin{verbatim}
#!/bin/bash

file=$1
nrows=$(awk 'END{print NR}' $file)
ncols=$(awk '{print NF}' $file | sort -u)
if [ $(echo $ncols | awk 'END{print NF}') -ne 1 ]
then
  echo "File lines do not have the same number of columns"
  exit 1
fi
# Debug
echo "File $file has $nrows lines and $ncols columns"
    \end{verbatim}
  \end{exampleblock}
\end{frame}


\begin{frame}
  \frametitle{Step}
  \framesubtitle{Exclude comments and finalization}

  \begin{exampleblock}{Task 4}
    Modify all parts of the previous script to exclude commented lines
   (starting with \code{\#})
  \end{exampleblock}

  \begin{alertblock}{Hint}
    You can use \code{grep}
  \end{alertblock}

  \begin{exampleblock}{Task 5}
    Ask the user which column he wants to perform the average on
  \end{exampleblock}

  \begin{alertblock}{Hint}
    \code{read -p}
  \end{alertblock}

  \begin{exampleblock}{Task 6}
    Complete the script, applying \code{binaver} to \code{\$1} with the
    known info
  \end{exampleblock}
\end{frame}


\begin{frame}[fragile=singleslide]
  \frametitle{Solution}
  \framesubtitle{Exclude comments and finalization}

  \begin{exampleblock}{Single file solution}
    \begin{verbatim}
#!/bin/bash

file=$1
nrows=$(grep -v "^#" $file | awk 'END{print NR}')
ncols=$(grep -v "^#" $file | awk '{print NF}' | sort -u)
if [ $(echo $ncols | awk 'END{print NF}') -ne 1 ]
then
  echo "File lines do not have the same number of columns"
  exit 1
fi
read -p "Enter column to be processed (1-$ncols):" target
# Debug
echo "binaver -r$nrows -c$ncols -k$target $file"
    \end{verbatim}
  \end{exampleblock}
\end{frame}


\begin{frame}
  \frametitle{Step}
  \framesubtitle{Multiple files, robustness}

  \begin{exampleblock}{Task 7}
    Extend the script to accept multiple files and act on all of them
  \end{exampleblock}

  \begin{alertblock}{Hint}
    Loop over arguments, \code{binaver} takes one at a time
  \end{alertblock}

  \begin{exampleblock}{Task 8}
    Increase robustness: skip a file if it doesn't exist
  \end{exampleblock}

  \begin{alertblock}{Hint}
    Use conditionals, the \code{else} branch can be empty
  \end{alertblock}
\end{frame}


\begin{frame}[fragile=singleslide]
  \frametitle{Solution}
  \framesubtitle{Multiple files, robustness}

  \begin{exampleblock}{Final complete solution}
    \small{
    \begin{verbatim}
#!/bin/bash

for file in $@
do
  if [ -f $file ]
  then
    nrows=$(grep -v "^#" $file | awk 'END{print NR}')
    ncols=$(grep -v "^#" $file | awk '{print NF}' | sort -u)
    if [ $(echo $ncols | awk 'END{print NF}') -eq 1 ]
    then
      echo "File $file fit for processing."
      read -p "Enter column to be processed (1-$ncols):" target
      # Debug
      echo "binaver -r$nrows -c$ncols -k$target $file"
    fi
  fi
done
    \end{verbatim}
    }
  \end{exampleblock}
\end{frame}


\begin{frame}[fragile=singleslide]
  \frametitle{Python}
  \framesubtitle{Whet your appetite with example!}

  \begin{exampleblock}{Read a dataset and plot}
    We want to read a scientific dataset, stored on disk
    in a scientific format data file, and plot the content.
    \begin{itemize}
       \item Access through community maintained API
    \end{itemize}
  \end{exampleblock}

  \begin{alertblock}{Environment}
    \verb|source /home/netapp-clima/users/ggiulian/m19.sh|
  \end{alertblock}
\end{frame}


\begin{frame}[fragile=singleslide]
  \frametitle{NetCDF Data format}
  \framesubtitle{xarray: labelled multi-dimensional arrays}

  \begin{exampleblock}{NetCDF Scientific data format}
     Standard for the climate and forecast is the netCDF.
    \begin{itemize}
       \item Use the xarray package
    \end{itemize}
  \end{exampleblock}

  \begin{exampleblock}{python code}
    \small{
    \begin{verbatim}
#!/usr/bin/env python3

from matplotlib import pyplot as plt
import xarray as xr

disk = "/home/esp-shared-a"
datapath = "Observations/BERKELEYEARTH/Gridded"
datafile = "Complete_TAVG_Daily_LatLong1_2010.nc"
ncfile = xr.open_dataset(disk+'/'+datapath+'/'+datafile)
ncfile.temperature
data = ncfile.temperature[0,:,:]
data.plot( )
plt.show( )
    \end{verbatim}
  }
  \end{exampleblock}
\end{frame}


\begin{frame}[fragile=singleslide]
  \frametitle{Seismic Data format}
  \framesubtitle{obspy : Python framework for processing seismological data}

  \begin{exampleblock}{obspy}
     Provide parsers for common file formats, clients to access data centers and
     seismological signal processing routines which allow the manipulation of
     seismological time series
  \end{exampleblock}

  \begin{exampleblock}{python code}
    \small{
    \begin{verbatim}
#!/usr/bin/env python3

from matplotlib import pyplot as plt
import obspy

site = "https://github.com/obspy/examples/raw/master"
file = "IU_ULN_2015-07-18T02.mseed"
data = obspy.read(site+'/'+file)
data.plot( )
plt.show( )
    \end{verbatim}
    }
  \end{exampleblock}

\end{frame}

\end{document}
